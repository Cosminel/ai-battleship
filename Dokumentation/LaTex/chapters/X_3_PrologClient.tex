\subsection{Client - Prolog} \label{sec:Prologclient}
	Die Aufgaben des Prolog-Clients sind in verschiedene Module unterteilt, um die verschiedenen Teile austauschbar zu halten. 
	Im Folgenden werden die Aufgaben und die Funktionsweise dieser Module erläutert.
	
	\todo{Gesamtübersicht - Bild - Modulzusammenspiel?}



\subsubsection{Hauptmodul}
	Das Hauptmodul \texttt{main.pl} wird beim starten des Clients aufgerufen. Das Modul initialisiert zunächst die Verbindung zum
	Spielserver. Anschließend werden Spielvorbereitungen über das Prädikat \texttt{initPrologClient/0} aus dem 
	Initialisierungsmodul getroffen (siehe Abschnitt \ref{sec:initModule}).
	
	Je nachdem ob der Client im Angriffs- oder Verteidigungsmodus startet, werden die Prädikate \texttt{attackFirst/0} oder
	\texttt{defendFirst/0} aufgerufen. Nach Empfang des Startsignals beginnt der Client dann mit einem Angriff oder Verteidigung.
	
	Ein Angriff verwendet die Prädikate \texttt{doAttack/2} und \texttt{attackResponse/3} des Angriffsmoduls (siehe Abschnitt
	\ref{sec:attackModule}).
	Das Prädikat \texttt{doAttack/2} liefert den Punkt auf dem gegnerischen Spielfeld der angegriffen werden soll.
	Die so erhaltenen X-Y-Koordinaten gibt das Hauptmodul an den Server weiter und wartet anschließend auf eine Antwort des 
	Gegners (über den Server). 
	Nach Erhalt dieser Antwort verwendet das Hauptmodul das Prädikat \texttt{attackResponse/3}, um die Antwort zu verarbeiten.
	Mit Abschluss der Verarbeitung ist der Angriff beendet.
	
	Die Verteidigung verwendet das Prädikat \texttt{doDefend/3} des Verteidigungsmoduls (siehe Abschnitt \ref{sec:defendModule}).
	Zu Beginn der Verteidigung wartet das Hauptmodul zunächst auf den Angriff des Gegners. Die so erhaltenen Koordinaten
	werden an das Prädikat \texttt{doDefend/3} übergeben. Als Resultat liefert dieses Prädikat die entsprechende Antwort für den
	Gegner. Das Hauptmodul sendet die Antwort gemäß dem Kommunikationsprotokoll an den Server. Damit ist die Verteidigung 
	beendet.
	
	Nach jedem Angriff überprüft der Client, ob er das Spiel gewonnen hat. Gleichermaßen überprüft er nach jeder Verteidigung, 
	ob er das Spiel verloren hat. Tritt einer der beiden Fälle in Kraft, so beendet sich der Client mit einer der beiden Ausgaben
	\textit{KI wins. End of game.} oder \textit{KI looses. End of game.}.
	

\subsubsection{Initialisierungsmodul} \label{sec:initModule}
	Die Aufgaben des Initialisierungsmoduls \texttt{initModule.pl} sind das initialisieren der Spielfelder, darunter auch
	das Platzieren der eigenen Schiffe. Hierzu werden die Prädikate \texttt{initMyField/0} und \texttt{initEnemyField/0} verwendet.
	
	Die verwendeten Spielfelder werden global in den dynamischen Prädikaten \texttt{myField/1} und \texttt{enemyField/1}
	gespeichert, um einen einfachen Zugriff von jedem Modul zu ermöglichen. 
	
	Das Initialisierungsmodul initialisiert außerdem die ebenfalls global gespeicherte Openlist, 
	die vom Strategiemodul verwendet wird (siehe Abschnitt \ref{sec:strategy}). 

\subsubsection{Modul zur Plazierung von Schiffen} \label{sec:initships}	
	\todo{TODO for Bogi}
	
\subsubsection{Verteidigungsmodul} \label{sec:defendModule}
	Die Aufgabe des Verteidigungsmoduls \texttt{defendModule.pl} ist es einen Angriff des Gegners zu verarbeiten.
	Zum einen muss dabei der Status des eigenen Feldes aktualisiert und zum anderen die Antwort für den Gegner bestimmt werden.
	
	Hat der Gegner ins Wasser geschossen, so ist keine Änderung des eigenen Feldes notwendig. Trifft der Gegner jedoch ein Schiff,
	so muss ermittelt werden, ob dieser Treffer das Schiff lediglich getroffen oder sogar versenkt hat. Außerdem ändert sich die Antwort
	für den Gegner, wenn das letzte Schiff versenkt wurde.
	
	Die Überprüfung, ob ein Schiff vollständig versenkt wurde, erfolgt über eine rekursive Überprüfung der benachbarten Felder.
	Dabei erhöht sich die Rekursionstiefe, wenn ein Nachbarfeld ebenfalls als getroffen markiert ist. 
	Da in den Regeln festgelegt ist, dass Schiffe sich nicht berühren dürfen, kann mit diesem Vorgehen festgestellt werden, ob ein Schiff
	vollständig versenkt wurde, oder sich unter den Nachbarn noch ungetroffene Teile befinden.
	
	\todo{Ne schöne Grafik?}
	
	
\subsubsection{Angriffsmodul} \label{sec:attackModule}


\subsubsection{Strategiemodul} \label{sec:strategy}


\subsubsection{Ausgabemodul}


