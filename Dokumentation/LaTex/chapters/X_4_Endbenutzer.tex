\section{Benutzungshinweise für Endbenutzer} \label{sec:Endbenutzer}

	Um das Spiel \textit{Schiffe-Versenken} zu spielen muss zunächst der Spielserver gestartet werden. Anschließend können, je nach
	Spielabsicht zwei der bereitgestellten Clients gestartet werden. Dabei ist es egal, ob zwei Java-Clients, zwei Prolog-Clients oder
	je ein Client von jedem Typ gestartet wird. 
	
	\todoin{Voraussetzung SWI Prolog und Java (Version bla oder höher) installiert}
	
\subsection{Starten des Servers}
	\todoin{Wie beschreibt man denn jetzt ein möglichst komfortables Starten der Java Geschichten?}

\subsection{Starten eines Java-Client}

\subsection{Starten eines Prolog-Client}
	Ein Prolog-Client kann über einen Doppelclick auf die Datei \texttt{start.bat} aus dem Ordner 
	\texttt{Battleship\textbackslash src\textbackslash prolog} 
	gestartet werden. Es öffnet sich eine Konsole, in der die Ausgaben des Clients (je nach Konfiguration) angezeigt werden.
	
	\subsubsection {Konfiguration des Prolog-Client}
		Standardmäßig ist der Prolog-Client für die Ausführung einer Runde konfiguriert. Außerdem werden die Spielzüge des Prolog-Clients auf
		der Kommandozeile ausgegeben. Sollen diese Einstellungen geändert werden so muss die Datei \texttt{main.pl} entsprechend angepasst werden.
		Das Prädikat \texttt{numberOfGames/1} kann zur Festlegung der Rundenanzahl und das Prädikat \texttt{verbose/1} zur Anpassung der Ausgabe 
		genutzt werden. Dabei steht der Wert \texttt{0} für keine Ausgabe, \texttt{1} für die Ausgabe in eine Datei und \texttt{2} für 
		die Ausgabe auf der Konsole.
	
