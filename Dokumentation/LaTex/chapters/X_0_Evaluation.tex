\section{Evaluation} \label{sec:Evaluation}

%\subsection{Testkonzept}

	Ziel der Evaluation war es, mögliche Fehlerzustände im implementierten Programm aufzudecken und außerdem zu belegen, dass das System
	wie erwartet funktioniert. 
	
	% White Box
	% \todoin{Entwicklertest, Testbarkeit der Software (Treiber,...), Positiv- und Negativtests,}
	Da die Verwendung strukturorientierter Testverfahren die Entwicklung eines Testrahmens bedeuten und dies den Rahmen des Projektes sprengen würde,
	wurde von der Verwendung dieser Verfahren abgesehen.  
	Stattdessen wurde der Einsatz verschiedener Black-Box Methoden vorgesehen.
	
	% Black Box
	% \subsection{Funktionaler Test}
	Um zu überprüfen ob das entwickelte Spiel die gewünschten Funktionen bietet, wurden funktionale Tests durchgeführt. 
	Als Spezifikation hierfür wurden die in Abschnitt \ref{sec:Spielregeln} beschriebenen Eigenschaften des Spiels \textit{Schiffe-Versenken}
	verwendet. 
	
	% nicht funktional
	Außerdem wurden auch nicht funktionale Testverfahren verwendet. In diesem Kontext wurde auch ein Langzeittest durchgeführt, bei dem zwei
	Prolog-Clients 100 Spiele gegeneinander spielen. Ziel dieses Tests war es eine hohe Anzahl von Spielen ohne Abbruch durchzuführen. 	

\subsection{Testfälle}
	Zur Umsetzung des beschriebenen Testkonzepts %(siehe Abschnitt \ref{sec:Evaluation}) 
	wurden konkrete Testfälle konzeptioniert und
	durchgeführt. Im Folgenden werden diese Testfälle, ihre Durchführung und die Testergebnisse beschrieben. 
	
	Für alle durchgeführten Testfälle des Prolog-Clients war es von Vorteil die Ausgabe in eine Textdatei umzuleiten (siehe Abschnitt \ref{ssub:ausgabeverhalten}), 
	um die Auswertung von Spielverläufen zu erleichtern.

	\subsubsection{Testfall 1, Testfall 2} % (fold)
	\label{ssub:testfall_1_testfall_2}
		Der erste Testfall untersucht die korrekte Initialisierung des Prolog-Clients. Insbesondere stellt die regelkonforme Platzierung der
		Schiffe einen wichtigen Bestandteil des Spiels dar. 
		Außerdem muss validiert werden, ob die geforderte Anzahl an Schiffen und die
		korrekten Schiffstypen auf dem Spielfeld zu finden sind. Dies stellt den zweiten Testfall dar.
	
		Zur Überprüfung der korrekten Platzierung, der richtigen Gesamtanzahl an Schiffen sowie der Anzahl der einzelnen Schiffstypen wurden 
		die Startaufstellungen von 100 KI-gegen-KI Spielen gespeichert und ausgewertet. Bei jedem KI-gegen-KI Spiel erzeugen die Prolog-Clients zwei 
		Aufstellungen (eine je KI-Spieler). Die so resultierenden 200 Aufstellungen wurden vom Entwicklerteam ausgewertet und auf Korrektheit bezüglich der 
		angegebenen Aspekte überprüft.
		
		\paragraph{Testergebnis} % (fold)
		\label{par:testergebnis}
			Die Überprüfung der 200 generierten Aufstellungen ergab folgendes Ergebnis:
			\begin{table}[H] % (fold)
				\centering
				\begin{tabular}{|p{.15\textwidth}|p{.14\textwidth}|p{.13\textwidth}|p{.15\textwidth}|p{.15\textwidth}|p{.11\textwidth}|} 
					\hline
					Geprüfte\newline Aufstellungen & Fehlerhafte Platzierung&Fehlerhafte Anzahl&Fehlerhafte Typen&Korrekte\newline Aufstellungen&Duplikate\\ 
					\hline\hline
					200 & 0 & 0 & 0 & 200 & 0\\
					\hline
				\end{tabular}
				\caption{Testresultat für Testfall 1 und Testfall 2}
				\label{tbl:tf1tf2}
			\end{table}
			% table tbl:tf1tf2 (end)
		% paragraph testergebnis (end)
		Das Testergebnis zeigt, dass die Routine zum Platzieren der Schiffe korrekt arbeitet. Beim Testen wurde gleichzeitig auch überprüft, 
		wie oft Duplikate durch die Routine erzeugt werden (also identische Platzierungen).
		Da die 200 überprüften Aufstellungen keine Duplikate enthalten, lässt schlussfolgern, dass der Einsatz der Prolog-Systemprädikate \texttt{random/1} und \texttt{randseq/3} 
		den gewünschten Effekt liefert (siehe Abschnitt \ref{sec:initships}).
		
		Die Liste der ausgewerteten Textdaten befindet sich in \texttt{Battle\-ship\textbackslash \-MyField\-\_200\-Test\-daten.txt}
	% subsubsection testfall_1_testfall_2 (end)
	\subsubsection{Testfall 3} % (fold)
	\label{ssub:testfall_3}
	
	Der dritte Testfall behandelt die Auswahl des nächsten anzugreifenden Spielfeldes. Im besonderen Fokus steht die Aufgabe der Verfolgung eines 
	getroffenen, aber noch nicht versenkten Schiffes.
	
	Hierfür wurde ein Spieler-gegen-KI Spiel durchgeführt und die Ausgaben der KI in eine Datei umgeleitet. Das Spiel wurde künstlich in die Länge 
	gezogen bis die KI gewann, indem der Spieler gezielt auf unbelegte Spielfelder der KI geschossen hat.
	Die mitgeschriebene Datei wurde anschließend analysiert. Hierbei wurden die von der KI gewählten Angriffskoordinaten mit den
	von der Strategie zu erwartenden Koordinaten verglichen (siehe Abschnitt \ref{sec:strategy}). Diese Analyse ist in Tabelle \ref{tbl:testfall3} nachzuvollziehen.
	\begin{table}[H] % (fold)
	\centering
	\begin{tabular}{|l|p{.2\textwidth}|p{.2\textwidth}|p{.15\textwidth}|p{.25\textwidth}|}
		\hline
		Nr.	&	beschossene Koordinate	&	verhalten Erwartungsgemäß	&	Antwort	&	Nächste zu erwartende Koordinate	\\
		\hline
		\hline
		0	&	-						& -								& -			& \emph{zufällig}						\\ \hline
		1	& 1 / 6						& \checkmark					& Wasser	& \emph{zufällig}						\\ \hline
		2	& 3 / 9						& \checkmark					& Treffer	& 2 / 9									\\ \hline
		3	& 2 / 9						& \checkmark					& Wasser	& 4 / 9									\\ \hline
		4	& 4 / 9						& \checkmark					& Treffer	& 5 / 9									\\ \hline
		5	& 5 / 9						& \checkmark					& \textbf{Versenkt}	& \emph{zufällig}						\\ \hline
		6	& 7 / 6						& \checkmark					& Wasser	& \emph{zufällig}						\\ \hline
		7	& 2 / 5						& \checkmark					& Wasser	& \emph{zufällig}						\\ \hline
		8	& 0 / 5						& \checkmark					& Wasser	& \emph{zufällig}						\\ \hline
		9	& 1 / 0						& \checkmark					& Wasser	& \emph{zufällig}						\\ \hline
		10	& 2 / 0						& \checkmark					& Treffer	& 3 / 0									\\ \hline
		11	& 3 / 0						& \checkmark					& Treffer	& 4 / 0									\\ \hline
		12	& 4 / 0						& \checkmark					& \textbf{Versenkt}	& \emph{zufällig}						\\ \hline
		13	& 8 / 1						& \checkmark					& Wasser	& \emph{zufällig}						\\ \hline
		14	& 2 / 3						& \checkmark					& Wasser	& \emph{zufällig}						\\ \hline
		15	& 3 / 6						& \checkmark					& Treffer	& 2 / 6									\\ \hline
		16	& 2 / 6						& \checkmark					& Wasser	& 4 / 6									\\ \hline
		17	& 4 / 6						& \checkmark					& Wasser	& 3 / 5									\\ \hline
		18	& 3 / 5						& \checkmark					& \textbf{Versenkt}	& \emph{zufällig}						\\ \hline
		19	& 1 / 9						& \checkmark					& Wasser	& \emph{zufällig}						\\ \hline
		20	& 5 / 2						& \checkmark					& Wasser	& \emph{zufällig}						\\ \hline
		21	& 2 / 4						& \checkmark					& Wasser	& \emph{zufällig}						\\ \hline
		22	& 6 / 1						& \checkmark					& Treffer	& 5 / 1									\\ \hline
		23	& 5 / 1						& \checkmark					& Wasser	& 7 / 1									\\ \hline
		24	& 7 / 1						& \checkmark					& Wasser	& 6 / 0									\\ \hline
		25	& 6 / 0						& \checkmark					& Wasser	& 6 / 2									\\ \hline
		26	& 6 / 2						& \checkmark					& Treffer	& 6 / 3									\\ \hline
		27	& 6 / 3						& \checkmark					& Treffer	& 6 / 4									\\ \hline
		28	& 6 / 4						& \checkmark					& \textbf{Versenkt}	& \emph{zufällig}						\\ \hline
		29	& 7 / 5						& \checkmark					& Wasser	& \emph{zufällig}						\\ \hline
		30	& 2 / 7						& \checkmark					& Wasser	& \emph{zufällig}						\\ \hline
		31	& 9 / 7						& \checkmark					& Wasser	& \emph{zufällig}						\\ \hline
		32	& 9 / 3						& \checkmark					& Treffer	& 8 / 3									\\ \hline
		33	& 8 / 3						& \checkmark					& Wasser	& 9 / 2									\\ \hline
		34	& 9 / 2						& \checkmark					& Treffer	& 9 / 4									\\ \hline
		35	& 9 / 4						& \checkmark					& Treffer	& 9 / 1									\\ \hline
		36	& 9 / 1						& \checkmark					& Wasser	& 9 / 5									\\ \hline
		37	& 9 / 5						& \checkmark					& Treffer	& 9 / 6									\\ \hline
		38	& 9 / 6						& \checkmark					& \textbf{Versenkt}	& \emph{\textbf{Spielende}}						\\ \hline
	\end{tabular}
	\caption{KI-Attacken entnommen aus \texttt{Gamelog1.txt}.}
	\label{tbl:testfall3}
\end{table}
% table tbl:testfall3 (end)
	
	\paragraph{Testergebnis} % (fold)
	\label{par:testergebnis}
	 Tabelle \ref{tbl:testfall3} stellt die von der KI gewählten Koordinaten, den erwarteten Koordinaten gegenüber. Es ist zu sehen, dass die KI zu jeder Zeit 
	die durch die Spielstrategie vorgegebenen Koordinaten auswählt.
	
	Die Liste der ausgewerteten Textdaten befindet sich in \texttt{Battle\-ship\textbackslash \-Out\-put\_1.txt}
	% paragraph testergebnis (end)
	% subsubsection testfall_3 (end)
	\subsection{Testfall 4} % (fold)
	\label{sub:testfall_4}
		Der vierte Testfall untersucht die korrekte Behandlung eines versenkten Schiffes. Insbesondere soll geprüft werden, ob die Umgebung in 
		einer 4er-Nachbarschaft
		mit Wasser aufgedeckt wird, da an diesen Stellen laut Regelbeschreibung (siehe Abschnitt \ref{sec:Spielregeln}) keine Schiffe platziert werden dürfen 
		und diese Felder somit für die nachfolgenden Züge uninteressant sind.
		
		Zur Testdurchführung wurde die für Testfall 3 (siehe Abschnitt \ref{ssub:testfall_3}) angefertigte Log-Datei an den Stellen ausgewertet, an denen ein 
		Schiff versenkt wurde. In Tabelle \ref{tbl:testfall3} sind diese Stellen fett hervorgehoben.
		An den betreffenden Stellen werden nun die Ausgaben des \texttt{enemyField} ausgewertet. Konkret wird das \texttt{enemyField} vor und 
		nach dem versenken eines Schiffes verglichen, um zu überprüfen ob die 4er-Nachbarschaft komplett mit Wasser aufgedeckt wird.
		\begin{longtable}{|m{.1\textwidth}|p{.26\textwidth}|p{.26\textwidth}|m{.3\textwidth}|}
		\hline
		Nr.	&	Vor versenken							&	nach Versenken							&	4er Nachbarschaft \newline korrekt aufgedeckt	\\
		\hline
		\hline
		5	& \texttt{\_  \_  \_  \_  \_  \_  \_  \_  \_  \_\newline
				\_  \_  \_  \_  \_  \_  \_  \_  \_  \_\newline
				\_  \_  \_  \_  \_  \_  \_  \_  \_  \_\newline
				\_  \_  \_  \_  \_  \_  \_  \_  \_  \_\newline
				\_  \_  \_  \_  \_  \_  \_  \_  \_  \_\newline
				\_  \_  \_  \_  \_  \_  \_  \_  \_  \_\newline
				\_  W  \_  \_  \_  \_  \_  \_  \_  \_\newline
				\_  \_  \_  \_  \_  \_  \_  \_  \_  \_\newline
				\_  \_  \_  \_  \_  \_  \_  \_  \_  \_\newline
				\_  \_  W  X  X  \_  \_  \_  \_  \_\newline
				}										&	\texttt{\_  \_  \_  \_  \_  \_  \_  \_  \_  \_\newline
															\_  \_  \_  \_  \_  \_  \_  \_  \_  \_\newline
															\_  \_  \_  \_  \_  \_  \_  \_  \_  \_\newline
															\_  \_  \_  \_  \_  \_  \_  \_  \_  \_\newline
															\_  \_  \_  \_  \_  \_  \_  \_  \_  \_\newline
															\_  \_  \_  \_  \_  \_  \_  \_  \_  \_\newline
															\_  W  \_  \_  \_  \_  \_  \_  \_  \_\newline
															\_  \_  \_  \_  \_  \_  \_  \_  \_  \_\newline
															\_  \_  \_  W  W  W  \_  \_  \_  \_\newline
															\_  \_  W  X  X  X  W  \_  \_  \_\newline	
															}										& \checkmark \\ \hline
		12	&	\texttt{\_  W  X  X  \_  \_  \_  \_  \_  \_\newline
				\_  \_  \_  \_  \_  \_  \_  \_  \_  \_\newline
				\_  \_  \_  \_  \_  \_  \_  \_  \_  \_\newline
				\_  \_  \_  \_  \_  \_  \_  \_  \_  \_\newline
				\_  \_  \_  \_  \_  \_  \_  \_  \_  \_\newline
				W  \_  W  \_  \_  \_  \_  \_  \_  \_\newline
				\_  W  \_  \_  \_  \_  \_  W  \_  \_\newline
				\_  \_  \_  \_  \_  \_  \_  \_  \_  \_\newline
				\_  \_  \_  W  W  W  \_  \_  \_  \_\newline
				\_  \_  W  X  X  X  W  \_  \_  \_\newline
				}  										&	\texttt{\_  W  X  X  X  W  \_  \_  \_  \_\newline
															\_  \_  W  W  W  \_  \_  \_  \_  \_\newline
															\_  \_  \_  \_  \_  \_  \_  \_  \_  \_\newline
															\_  \_  \_  \_  \_  \_  \_  \_  \_  \_\newline
															\_  \_  \_  \_  \_  \_  \_  \_  \_  \_\newline
															W  \_  W  \_  \_  \_  \_  \_  \_  \_\newline
															\_  W  \_  \_  \_  \_  \_  W  \_  \_\newline
															\_  \_  \_  \_  \_  \_  \_  \_  \_  \_\newline
															\_  \_  \_  W  W  W  \_  \_  \_  \_\newline
															\_  \_  W  X  X  X  W  \_  \_  \_\newline
															}										& \checkmark \\ \hline
		18	&	\texttt{\_  W  X  X  X  W  \_  \_  \_  \_\newline
				\_  \_  W  W  W  \_  \_  \_  W  \_\newline
				\_  \_  \_  \_  \_  \_  \_  \_  \_  \_\newline
				\_  \_  W  \_  \_  \_  \_  \_  \_  \_\newline
				\_  \_  \_  \_  \_  \_  \_  \_  \_  \_\newline
				W  \_  W  \_  \_  \_  \_  \_  \_  \_\newline
				\_  W  W  X  W  \_  \_  W  \_  \_\newline
				\_  \_  \_  \_  \_  \_  \_  \_  \_  \_\newline
				\_  \_  \_  W  W  W  \_  \_  \_  \_\newline
				\_  \_  W  X  X  X  W  \_  \_  \_\newline
				}  										&	\texttt{\_  W  X  X  X  W  \_  \_  \_  \_\newline
															\_  \_  W  W  W  \_  \_  \_  W  \_\newline
															\_  \_  \_  \_  \_  \_  \_  \_  \_  \_\newline
															\_  \_  W  \_  \_  \_  \_  \_  \_  \_\newline
															\_  \_  \_  W  \_  \_  \_  \_  \_  \_\newline
															W  \_  W  X  W  \_  \_  \_  \_  \_\newline
															\_  W  W  X  W  \_  \_  W  \_  \_\newline
															\_  \_  \_  W  \_  \_  \_  \_  \_  \_\newline
															\_  \_  \_  W  W  W  \_  \_  \_  \_\newline
															\_  \_  W  X  X  X  W  \_  \_  \_\newline
															}										& \checkmark \\ \hline
		28	&	\texttt{\_  W  X  X  X  W  W  \_  \_  \_\newline
				\_  \_  W  W  W  W  X  W  W  \_\newline
				\_  \_  \_  \_  \_  W  X  \_  \_  \_\newline
				\_  \_  W  \_  \_  \_  X  \_  \_  \_\newline
				\_  \_  W  W  \_  \_  \_  \_  \_  \_\newline
				W  \_  W  X  W  \_  \_  \_  \_  \_\newline
				\_  W  W  X  W  \_  \_  W  \_  \_\newline
				\_  \_  \_  W  \_  \_  \_  \_  \_  \_\newline
				\_  \_  \_  W  W  W  \_  \_  \_  \_\newline
				\_  W  W  X  X  X  W  \_  \_  \_\newline
				}  										&	\texttt{\_  W  X  X  X  W  W  \_  \_  \_\newline
															\_  \_  W  W  W  W  X  W  W  \_\newline
															\_  \_  \_  \_  \_  W  X  W  \_  \_\newline
															\_  \_  W  \_  \_  W  X  W  \_  \_\newline
															\_  \_  W  W  \_  W  X  W  \_  \_\newline
															W  \_  W  X  W  \_  W  \_  \_  \_\newline
															\_  W  W  X  W  \_  \_  W  \_  \_\newline
															\_  \_  \_  W  \_  \_  \_  \_  \_  \_\newline
															\_  \_  \_  W  W  W  \_  \_  \_  \_\newline
															\_  W  W  X  X  X  W  \_  \_  \_\newline
															}										& \checkmark \\ \hline
		28	&	\texttt{\_  W  X  X  X  W  W  \_  \_  \_\newline
				\_  \_  W  W  W  W  X  W  W  W\newline
				\_  \_  \_  \_  \_  W  X  W  \_  X\newline
				\_  \_  W  \_  \_  W  X  W  W  X\newline
				\_  \_  W  W  \_  W  X  W  \_  X\newline
				W  \_  W  X  W  \_  W  W  \_  X\newline
				\_  W  W  X  W  \_  \_  W  \_  \_\newline
				\_  \_  W  W  \_  \_  \_  \_  \_  W\newline
				\_  \_  \_  W  W  W  \_  \_  \_  \_\newline
				\_  W  W  X  X  X  W  \_  \_  \_\newline
				}  										& \emph{Spielende}									& \checkmark \\ \hline
	\caption{Ausgabe des \texttt{enemyField} aus \texttt{Gamelog1.txt}.}
	\label{tbl:testfall4}
	\end{longtable}
		Zum besseren erkennen der Muster, wurden die Ausgaben des Ausgabemoduls nachträglich bearbeitet. Die Markierung für unbekannte Felder \texttt{U} wurde 
		durch \texttt{\_} ersetzt.
		\paragraph{Testergebnis} % (fold)
		\label{par:testergebnis}
			Wie in Tabelle \ref{tbl:testfall4} zu erkennen, wird nach jedem versenken eines Schiffes die komplette \emph{4er-Nachbarschaft} mit Wasser 
			aufgedeckt so wie es in im Abschnitt zur Spielstrategie (\ref{sec:strategy}) beschrieben ist.

			Die Liste der Ausgewerteten Textdaten befindet sich in \texttt{Battle\-ship\textbackslash \-Out\-put\_1.txt}
		% paragraph testergebnis (end)
	% subsection testfall_4 (end)
    \subsubsection{Unittest: CPlayingFieldController}
Die Controllerklasse \texttt{CPlayingFieldController} des Java-Clients wurde einem Unittest unterzogen.
Diese Klasse bildet das zentrale Element der Spiellogik.
Das Ziel dieses Tests ist die Sicherstellung der korrekten Statuswechsel, sowie die korrekte Repräsentation der Spielinformationen.

Jedoch ist eine Untersuchung der einzelnen Methoden in diesem Fall nicht zielführend, da die internen Zustände und Variablen nur durch den Nachrichtenaustausch indirekt manipuliert werden können.
Außerdem ist zu beachten, dass die Schiffsplatzierung zufällig erfolgt, sodass diese Komponente nicht vorhergesagt und somit geprüft werden kann.

Die Testsequenz entspricht den ersten Zügen eines angehenden Spiels. 
Sie unterteilt sich in die primär in die Phase der Initialisierung und die des laufenden Spiels.

Im Rahmen des Tests werden zwei Instanzen der Klasse \texttt{CPlayingFieldController} erzeugt, die sich mit dem Spieleserver verbinden.
Unmittelbar nachdem sich ein Client verbunden hat, geht dieser in die Initialisierungphase über, was anhand des Unittests bestätigt wurde.
Außerdem konnte ebenfalls bestätigt werden, dass der erste Spielteilnehmer erwartungemäß im Verteidigungszustand startet.

Im weiteren Verlauf der Testfolge verbindet sich ein zweiter Client.
Erwartungsgemäß konnte verifiziert werden, dass dieser zum Einen den ersten Angriff ausführt und zum Anderen in den \emph{RUNNING}-Status gewechselt ist.

In der zweiten Phase des Tests werden abwechselnd Angriffe gestartet und es wird geprüft, ob sich die internen Spielfelder den eingehenden Informationen entsprechend geändert haben.
Die korrekte Darstellung des gegnerischen Spielfelds von \emph{UNKNOWN} zu \emph{WATER} bzw. \emph{HIT} konnte bestätigt werden.

Nach zwei weiteren Spielzügen konnte der korrekte Wechsel vom Angriffs- in den Verteidigungszustand und vice versa verifiziert werden.

Da die Schiffe zufällig platziert werden, kann eine Evaluation des Spielendes nicht automatisch erfolgen.
Aus diesem Grund wurde händisch ein Spiel Java-Client vs. Java-Client ausgetragen und geprüft, ob nach dem letzten versenkten Schiff der entsprechende Statuswechsel erfolgte.
Auch dieser Test konnte erfolgreich bestätigt werden.