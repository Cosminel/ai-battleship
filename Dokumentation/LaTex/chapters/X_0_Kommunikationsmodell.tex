\section{Kommunikationsprotokoll}
\label{sec:Kommunikationsmodell}

Die Kommunikation zweier Spieler erfolgt auf Basis von Textnachrichten, den sogenannten Kommandos, die einer fest definierten Codierung genügen.
Das Ziel dieses Kapitels ist die Erläuterung des Nachrichtenaufbaus und deren Interpretation.

Jedes Kommando, das es zu interpretieren gilt, ist aus Gründen der Prologkompatibilität mit runden Klammern umgeben und endet mit einem "'."'.
Da zudem stets ganze Zeilen verarbeitet werden sollen, wird der Nachricht das nicht druckbare Zeichen "'\textbackslash n"' angehängt.
Damit wird signalisiert, dass die gesamte Nachricht übermittelt wurde.

Innerhalb der umschließenden runden Klammern werden zwei Angaben erwartet.
Die erste Angabe wird als \emph{Opcode} bezeichnet und dient der eindeutigen Bestimmung des Kommandotyps.
Die zweite Angabe des Kommandos ist eine, von eckigen Klammern umgebene Parameterliste und kann null bis drei, durch Kommata getrennte, nummerische Elemente beinhalten.
Die Anzahl der erwarteten Parameter richtet sich nach dem übermittelten Opcode.

Die formale Beschreibung des Kommandoaufbaus liegt im Folgenden in Form von regulären Ausdrücken vor.

\begin{align*}
	COMMAND &:= \text{\textbackslash}(OPCODE,LIST\text{\textbackslash}) \text{\textbackslash}. NEWLINE \\
	LIST &:= \text{\textbackslash}[ [PARAMS]? \text{\textbackslash}] \\
	NEWLINE &:= \text{\textbackslash \textbackslash} r \\
	OPCODE &:= [1-5] \\
	PARAMS &:= [0-9]+[,[0-9]+]\{0,2\}
\end{align*}

Der Spielablauf durchläuft mehrere Zustände, die durch das Kommunikationsprotokoll abgedeckt werden müssen.
So startet jeder Spieler in einem Initialisierungszustand.
In diesem Zustand wird festgelegt, welcher der beiden Spieler den ersten Angriff ausführt.

\todoin{Da wird plötzlich vom Server angefangen, ohne jemals erwähnt zu haben, dass wir einen haben! Den Absatz vllt in eine Server - section packen?}
Stellt der Server eine gültige Anzahl an verbundenen Spielteilnehmern fest, sendet dieser ein Startsignal an alle Teilnehmer und die Clients wechseln ihrerseits vom Initialisierungszustand in den Angriffs-, bzw. Verteidigungszustand.
Weitere Nachrichten die es zu übertragen gilt, sind zum Einen der Angriff auf eine Feldkoordinate, sowie zum Anderen das Ergebnis des Angriffs.

Die Abbildung der genannten Aktionen in Opcodes, sowie die erwarteten Parameter können der Tabelle \ref{tbl:Kommunikationsprotokoll} entnommen werden.

\begin{table}[H]
	\centering
	\begin{tabular}{|c|l|c|c|c|}
		\hline
		\textbf{Opcode}	&	\textbf{Bedeutung}	&	\textbf{Param 1}	&	\textbf{Param 2}	&	\textbf{Param 3} \\
		\hline
		\hline
		1				&	Angriff				&	X						&	Y				&	-	\\
		\hline
		2				&	Ergebnis des Angriffs	&	X					&	Y				&	Ergebnis	\\
		\hline
		3				&	Spieler startet im Verteidigungszustand	&	-	&	-				&	-	\\
		\hline
		4				&	Spieler startet im Angriffszustand	&	-		&	-				&	-	\\
		\hline
		5				&	Startsignal			&	-						&	-				&	-	\\
		\hline
	\end{tabular}
	\caption{Kommunikationsprotokoll}
	\label{tbl:Kommunikationsprotokoll}
\end{table}

Das Ergebnis eines Angriffs wird ebenfalls kodiert übertragen und kann den Zustand des Clients beeinflussen.
Im Regelfall sendet bzw. empfängt dieser die Ereignisse "'Wasser"', "'Schiff wurde getroffen"', oder "'Schiff wurde versenkt"'.
Des Weiteren kann als Reaktion auf einen Angriff die Nachricht eintreffen, dass das letzte Schiff versenkt wurde.
In diesem Fall wechseln die Clientzustände in Abhängigkeit vom Sender und Empfänder dieser Nachricht in "'gewonnen"' bzw. "'verloren"'.

Eine Auflistung der möglichen Ergebnisse eines Angriffs ist in Tabelle \ref{tbl:Ergebniskodierung} gegeben.

\begin{table}[H]
	\centering
	\begin{tabular}{|c|l|c|c|c|}
		\hline
		\textbf{Code}	&	\textbf{Bedeutung} \\
		\hline
		\hline
		1				&	Wasser	\\
		\hline
		2				&	Schiff wurde getroffen	\\
		\hline
		3				&	Schiff wurde getroffen und versenkt	\\
		\hline
		4				&	Letztes Schiff wurde versenkt. Das Spiel ist beendet	\\
		\hline
	\end{tabular}
	\caption{Ergebniskodierung}
	\label{tbl:Ergebniskodierung}
\end{table}
