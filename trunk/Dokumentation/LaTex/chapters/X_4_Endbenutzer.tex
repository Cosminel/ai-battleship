\section{Benutzungshinweise für Endbenutzer} \label{sec:Endbenutzer}
	\todoin {Gut so?! -SB}
	Um das Spiel \textit{Schiffe-Versenken} zu spielen muss zunächst der Spielserver gestartet werden. Anschließend können, je nach
	Spielabsicht zwei der bereitgestellten Clients gestartet werden. Dabei ist es egal, ob zwei Java-Clients, zwei Prolog-Clients oder
	je ein Client von jedem Typ gestartet wird. 
	
	\subsection{Systemvoraussetzungen} % (fold)
	\label{sub:systemvoraussetzungen}
		Um ein Fehlerfreies starten des Servers und der Clients gewährleisten zu können, ist ein Installiertes Java-Runtime-Environment 
		der Version 1.6 erforderlich. Dieses kann ggf. unter \url{http://www.java.com/de/download}(stand: 16.01.2011) heruntergeladen werden.
		\newline
		Des weiteren muss SWI-Prolog in Version 5.10.1 auf dem System installiert sein, um die KI-Komponenten ausführen zu können. SWI-Prolog 
		kann ggf. unter \url{http://www.swi-prolog.org/Download.html} bezogen werden.
	% subsection systemvoraussetzungen (end)
%	\todoin{Voraussetzung SWI Prolog und Java (Version bla oder höher) installiert}
	
\subsection{Starten des Servers}
%	\todoin{Wie beschreibt man denn jetzt ein möglichst komfortables Starten der Java Geschichten?}
	Der zum spielen benötigte Server, \texttt{Battleship\_Server.jar}, kann entweder selbst gestartet werden, oder 
	durch eines der beiliegenden Start-Skripte.
	Ist es erforderlich, den Server eigenhändig zu starten, kann dies über den Kommandozeilenbefehl:\newline
	\texttt{java -jar Battleship\_Server.jar}\newline
	im, den Server beinhaltenden, Ordner getan werden.

\subsection{Starten eines Java-Client}
	Der Client, welcher das eigene Spielfeld abbildet, \texttt{Battleship\_Client.jar}, kann, analog zum Server, entweder selbst gestartet werden, oder 
	durch eines der beiliegenden Start-Skripte.
	Ist es erforderlich, den Client eigenhändig zu starten, kann dies über den Kommandozeilenbefehl:\newline
	\texttt{java -jar Battleship\_Client.jar}\newline
	im, den Client beinhaltenden, Ordner getan werden.

\subsection{Starten eines Prolog-Client}
	Ein Prolog-Client kann über einen Doppelclick auf die Datei \texttt{start\_prolog\_Windows} aus dem Ordner 
	\texttt{Battleship} 
	gestartet werden. Es öffnet sich eine Konsole, in der die Ausgaben des Clients (je nach Konfiguration) angezeigt werden.
	
	\subsubsection {Konfiguration des Prolog-Client}
		Standardmäßig ist der Prolog-Client für die Ausführung einer Runde konfiguriert. Außerdem werden die Spielzüge des Prolog-Clients auf
		der Kommandozeile ausgegeben. Sollen diese Einstellungen geändert werden so muss die Datei \texttt{main.pl} entsprechend angepasst werden.
		Das Prädikat \texttt{numberOfGames/1} kann zur Festlegung der Rundenanzahl und das Prädikat \texttt{verbose/1} zur Anpassung der Ausgabe 
		genutzt werden. Dabei steht der Wert \texttt{0} für keine Ausgabe, \texttt{1} für die Ausgabe in eine Datei und \texttt{2} für 
		die Ausgabe auf der Konsole.
\subsection{Startskripte} % (fold)
\label{sub:startskripte}
	Die beiliegenden Startskript starten alle zum Spielen erforderliche Komponenten. Es sind Startskript für drei Spielmodi vorhanden:
	\begin{itemize}
		\item Menschlicher Spieler gegen KI-Spieler: \texttt{start\_Human-KI-Match\_\emph{OS}.bat}
		\item KI-Spieler gegen KI-Spieler: \texttt{start\_KI-KI-Match\_\emph{OS}.bat}
		\item Menschlicher Spieler gegen Menschlischen Spieler: \texttt{start\_Human-Human-Match.bat}
	\end{itemize}
	Anmerkung: \texttt{\emph{OS}} ist durch "Windows" oder "Unix" zu ersetzen, je nach Betriebssystem.
	
	Die Benutzung dieser Startskripts führt dazu, dass alle benötigten Komponenten auf der Konsole gestartet werden. Dies führt dazu, dass die Konsole 
	auf der das Startskript ausgeführt wird, mit den Ausgaben aller drei Komponenten beschrieben wird. Für Debugging-Zwecke sollte also auf die Verwendung dieser 
	Skripte verzichtet werden.
% subsection startskripte (end)
