\section{Einleitung}
\label{sec:Einleitung}
	\todoin {Sibille, bitte Rechtschreibung prüfen :-/ \newline Reicht das so!?}
	Im Wahlpflichtfach "'Künstliche Intelligenz"' wurde während der Vorlesungen, den Vorlesungsbegleitenden Beispielen 
	sowie in den Laboraufgaben ein solides Verständnis und Wissen über die Implementierung einer künstlichen Intelligenz 
	mithilfe der Programmiersprache "'Prolog"' erarbeitet. Zum Ende der Veranstaltung soll das angeeignete Wissen nun in 
	Form eines Projekts in Gruppenarbeit dargestellt und weiter vertieft werden.
	
	Die vorliegende Dokumentation beschreibt die Entwicklung des Spiels "'Schiffe-Versenken"' ("Battleship"), welches 
	mithilfe der Programmiersprachen Java und Prolog implementiert wird.
	
	Ziel des Projekts ist es das Spiel gegen den "'Computer"' spielen zu können. Hierfür muss das Spiel mit allen Spielregeln 
	in Prolog abgebildet werden. Ebenso muss eine Spielstrategie für den Computerspieler konzeptioniert und implementiert werden, 
	welche dafür sorgt das die künstliche Intelligenz Siegesorientiert spielt. 
	Eine Benutzungsfreundliche Oberfläche für menlische Spieler wird durch ein Java-Programm realisiert.
	
	\subsection{Anforderungen} % (fold)
	\label{sub:anforderungen}
		Von den Gruppenmitgliedern wurden die Folgenden Anforderungen für die Software definiert:
		\begin{itemize}
			\item Das Prolog Programm kommuniziert über Sockets mit der Java-Benutzerschnittstelle.
			\item Prolog-Implementierungen von Modulen zum Regeln des/der:
			\begin{itemize}
				\item generellen Spielablaufs,
				\item platzieren von Schiffen auf dem Spielfeld,
				\item Antworten auf "'Angriffe"',
				\item "'Angreifen"' des gegenspielers nach einer 
				\item Siegstrategie.
			\end{itemize}
			\item Es soll möglich sein, selbst gegen die künstliche Intelligenz zu spielen.
			\item Es soll möglich sein, dass zwei Instanzen der künstlichen Intelligenz eine bestimmte Anzahl von spielen gegeneinander 
			austragen.
		\end{itemize}
		Neben diesen Anforderungen ist eine weitere, spezielle, Anforderung definiert:
		
		Da eine weitere Gruppe von Studenten des Moduls ebenfalls das Spiel "'Schiffe-Versenken"' als Projektthema wählten, soll 
		es möglich sein, dass die künstlichen Intelligenzen beider Gruppen gegeneinander antreten können. 
		Hierfür ist die Abstimmung eines Protokolls zur Kommunikation mit der anderen Gruppe notwendig.
	% subsection anforderungen (end)