\section{Einleitung}
\label{sec:Einleitung}
	\todoin {Vic - Reicht das so!?}
	Im Wahlpflichtfach \textit{Künstliche Intelligenz} wurde während der Vorlesungen und den vorlesungsbegleitenden Beispielen 
	ein solides Verständnis und Wissen über die Implementierung einer künstlichen Intelligenz 
	in Prolog erarbeitet. Zum Ende der Veranstaltung soll das angeeignete Wissen nun in 
	Form eines Projektes angewendet und vertieft werden.
	
	Die vorliegende Dokumentation beschreibt die Implementierung des Spiels Schiffe-Versenken (Battleship)
	in Java und Prolog.
	
	Ziel des Projektes ist es \textit{Schiffe-Versenken} gegen "'den Computer"' spielen zu können. Hierfür muss das Spiel mit allen Spielregeln 
	in Prolog abgebildet werden. Ebenso muss eine Spielstrategie für den Computerspieler konzeptioniert und implementiert werden, 
	welche dafür sorgt das die künstliche Intelligenz Siegesorientiert spielt. 
	Eine Benutzungsfreundliche Oberfläche für menlische Spieler wird durch ein Java GUI realisiert.
	
	\todoin{Vic - hier auch den Server erwähnen?}
	
	\subsection{Anforderungen} % (fold)
	\label{sub:anforderungen}
		Die Software unterliegt den folgenden Anforderungen:
		\begin{itemize}
			\item Es gelten die allgemeinen Regeln des Spiels \textit{Schiffe-Versenken}, Besonderheiten werdnen in Abschnitt \ref{sec:Spielregeln} erläutert.
			\item Der Prolog Client kommuniziert über Sockets mit dem zu entwickelnden Java Server.
			\item Benötigte Module für den Prolog Client zum Regeln:
			\begin{itemize}
				\item des generellen Spielablaufs,
				\item der Platzierung von Schiffen auf dem Spielfeld,
				\item der Antworten auf gegnerische Angriffe,
				\item der eigenen Angriffe nach einer 
				\item Siegstrategie.
			\end{itemize}
			\item Es soll möglich sein, selbst gegen die künstliche Intelligenz zu spielen.
			\item Es soll möglich sein, dass zwei Instanzen der künstlichen Intelligenz eine bestimmte Anzahl von Spielen gegen einander 
			austragen.
		\end{itemize}
		Neben diesen Anforderungen ist eine weitere, spezielle, Anforderung definiert:
		
		Da eine weitere Gruppe von Studenten dieses Moduls ebenfalls das Spiel \textit{Schiffe-Versenken} als Projektthema wählten, soll 
		es möglich sein, dass die künstlichen Intelligenzen beider Gruppen gegeneinander antreten können. 
		Hierfür wurde das, in Abschnitt \ref{sec:Kommunikationsmodell} dokumentierte Kommunikationsprotokoll unter beiden Gruppen vereinbart.
	% subsection anforderungen (end)